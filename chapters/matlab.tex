\chapter{MATLAB}

\section{将内置函数背下来}

在MATLAB中工作时,很多操作其实都是有内置函数的,对MATLAB不熟悉就用不成,然后就“曲线救国”了,效率不高且不说,关键自己实现很费脑!这里收集一些数值计算工作中极其常用的函数。

\mintinline{Matlab}{meshgrid}

\section{如何遍历当前文件夹及其子文件夹中的全部文件?}

假设现在我们有这样一个文件夹A,它含有一些文件和子文件夹B、C、D......,这些子文件夹又包含若干层子文件夹。我们需要将这个父文件夹(A)及其子文件夹(B、C、D......)和孙文件夹中的所有文件名和其路径取出来。

如果你用的是MATLAB 2016b及更新的版本,那真的太棒了!\mintinline{Matlab}{dir()}函数已经支持遍历搜索了。尝试敲入:

\begin{minted}{Matlab}
dir_data = dir('**/*');
dir_data([dir_data.isdir]) = [];  % 去除所有.和..文件夹
\end{minted}

这将会返回一个包含文件信息的struct,现在你可以任意操作这些struct了,随意拼接路径。解放大脑,哦也!方便归方便,但是,一来肯定有大多数人使用的是MATLAB 2016b之前的版本,二来,解放大脑意味着我失去了一次独立思考的机会。

\subsection*{思考}

对于实现方法\footnote{思路来源:\href{https://stackoverflow.com/questions/2652630/how-to-get-all-files-under-a-specific-directory-in-matlab}{How to get all files under a specific directory in MATLAB?}},多层次的遍历,我第一时间想到的是递归。然后就是数据的存储了,\mintinline{Matlab}{dir()}函数返回的是一个struct,这个数据结构储存有文件的信息,我们要充分利用这个数据结构。所以现在思路是,写一个递归函数,这个函数返回包含所有文件信息的struct。

这个函数应对先处理父文件夹,获取文件和子文件夹,然后储存文件信息,同时去除子文件夹中的`.'和`..'这两个特殊文件夹。我们对获取的子文件夹再次调用该函数,并储存文件信息。如此,利用递归获取子子孙孙无穷尽文件夹的信息\footnote{其实这并不可能,因为递归是有栈高度限制的,调用函数压入栈,返回函数弹出栈,如果文件夹层次太深,一直压栈就会到达栈溢出警告的极限,例如Python的栈往往是100层,我想MATLAB的栈也大致如此,不会太高},最后函数返回存储有所有文件信息的struct。现在,你可以对这个结构体做你想做的事情。

\subsection*{解}

MATLAB 2016b以上的版本我们可以用函数返回struct,这个数据结构包含[folder, name, date, bytes, isdir, datenum]六个字段的信息,我们可以按自己意愿使用folder和name拼接出文件的完整路径。

\inputminted[firstline=1]{Matlab}{code/matlab/get_all_file_name_R2016b_newer.m}

MATLAB 2016a及之前的版本dir struct信息并不包含folder,如果返回struct,将只有文件的[name, date, bytes, isdir, datenum]五个字段的信息,所以我们并不能根据函数返回的struct拼接出文件完整路径,\textbf{我们需要自己将路径拼接成一个cell,然后使用函数返回cell}。

\inputminted{Matlab}{code/matlab/get_all_file_name_R2016a_older.m}

\subsection*{总结}
\mintinline{Matlab}{dir()}函数遍历整个F盘共2万余文件文件大约需要1.555823s。我们实现的递归函数遍历F盘文件大约需要3.703009s。慢是慢了点,但我们成功运用了递归解决问题,不是吗?

\section{如何按自然顺序排序字符串?}

通常,我们会遇到处理一系列文件名有规律的文件的情况,比如:a1.txt、a2.txt ...... a100.txt。但是,当读取文件名到一个cell里后,我们发现文件名往往是乱序排列的,甚至当你使用\mintinline{Matlab}{sort}函数后,排序也不会改变。搜索了一下,在Mathworks File Exchange网站找到了一个自然排序的函数\footnote{\url{https://cn.mathworks.com/matlabcentral/fileexchange/34464-customizable-natural-order-sort}},感谢作者Stephen Cobeldick。效果如下:

\begin{figure}[h]
    \centering
    \begin{minipage}{0.45\textwidth}
        \centering
        \includegraphics[height=6cm]{disordered_file_name}
        \caption{乱序的文件名}
    \end{minipage}
    \begin{minipage}{0.45\textwidth}
        \centering
        \includegraphics[height=6cm]{ordered_file_name}
        \caption{排序后自然顺序的文件名}
    \end{minipage}
\end{figure}

\section{如何隔行取数据?}

闭上眼睛,想象现在有这样一个数组\mintinline{Matlab}{[1, 2, 3, 4, 5, 6, 7, 8, 9, 10]},我们要隔一列取一个数据,或者隔两列取一个数据。得益于MATLAB的向量化编程,我们可以很方便的做到,

\begin{minted}{Matlab}
mat_a = [1, 2, 3, 4, 5, 6, 7, 8, 9, 10];
mat_b = mat_a(:, 1:2:length(mat_a));
\end{minted}

如果你用循环,那么你的代码就不优雅,另,向量化操作比循环快,大型数组优势明显。以上。

\section{如何在遍历数组的同时删除被遍历过的元素?}

闭上眼睛,想象现在有这样一个数组\mintinline{Matlab}{[1, 2, 3, 4, 5, 6, 7, 8, 9, 10]},我们需要边遍历元素边删除元素。实现方法和Python章节方法一致。

\begin{minted}{Matlab}
mat_a = [1, 2, 3, 4, 5, 6, 7, 8, 9, 10];

while ~isempty(mat_a)
    fprintf("The element being traversed is %d\n", mat_a(1));
    mat_a(1) = [];
    disp(mat_a);
end
\end{minted}

\section{再谈向量化操作}

今天又碰到一个数组操作的问题,同样,如果用一般的方法来解决,代码是很冗长的,向量化操作再次助我一臂之力。

有一个2列的数组\mintinline{Matlab}{all_data},第一列有正有负,我们称第一列为$ x $,第二列为$ y $。现在需要索引$ x>0 $时对应的$ [x, y] $为一个新的数组\mintinline{Matlab}{a}。并且需要从\mintinline{Matlab}{a}中返回$ y=\min(y) $时所对应的数组$ [x_0, y_0] $。

\begin{minted}{Matlab}
... ...
-1.44319267634370e-06 9.80637785912817e-06
-1.68967863180042e-07 9.73806551956721e-06
-6.45218837777561e-07 9.75074561060079e-06
6.28923217787410e-07 9.75037059307950e-06
1.54045071473931e-07 9.73772289244816e-06
1.42591401762642e-06 9.80510552313044e-06
... ...
\end{minted}

先谈向量化获取数组\mintinline{Matlab}{a},利用逻辑索引,保证$ x $全大于0,并取出1、2两列;然后利用\mintinline{Matlab}{find}获取$ y=\min(y) $的行索引;最后利用索引轻松找到需要的数据。可能看起来比较难理解,但是此时再在外面套文件操作的循环等循环操作是不是清晰多了。

\begin{minted}{Matlab}
a = all_data(all_data(:, 1) > 0, 1:2);

[r, ~] = find(a(:, 2) == min(a(:, 2)));
what_is_i_need = a(r, c);
\end{minted}

\begin{minted}{Matlab}
clear
clc

file_name_struct = dir('./0518*.txt');
file_name = {file_name_struct(:).name};
file_name = natsort(file_name);
what_is_i_need = [];

for file_num = 1:length(file_name)
    all_data = load(file_name{file_num});
    a = all_data(all_data(:, 1) > 0, 1:2);
    
    [r, ~] = find(a(:, 2) == min(a(:, 2)));  % 
    what_is_i_need = [what_is_i_need; a(r, 1), a(r, 2)];
end

plot(what_is_i_need(:,2))
\end{minted}

再来记录一个问题:循环操作里面有一个的2列数组\mintinline{Matlab}{all_data},每次循环取第一列中与0.5最接近的数据和对应的列,所以该数组大小会不断变,设其维度为$ n\times 2 $。如果$ n<2 $,我们将数据置为0并保存到一个新数组里面去;如果$ n>=2 $,保存其最小值和最大值到新数组里面去。同样,利用向量化操作最大程度减少代码量。

\begin{minted}{Matlab}
pos = [];
for i = 1:length(time)
    % [x, phi]
    all_data = load(strcat(mph_file, '\', num2str(i), '.txt'));
    all_data = all_data(abs(all_data(:, 2)-0.5) < 0.01, :);
    [r, c] = size(all_data);
    if r == 0 || r == 1
        x_min = 0;
        x_max = 0;
        phi = 0;
        pos = [pos; time(i), x_min, phi; time(i), x_max, phi];
    else
        x_min = all_data(all_data(:, 1) == min(all_data(:, 1)));
        x_max = all_data(all_data(:, 1) == max(all_data(:, 1)));
        [r, ~] = find(all_data == x_min);
        phi_min = all_data(r, 2);
        [r, ~] = find(all_data == x_max);
        phi_max = all_data(r, 2);
        pos = [pos; time(i), x_min, phi_min; time(i), x_max, phi_max];
    end
end
\end{minted}

\section{文件读取}

\mintinline{Matlab}{csvread}适合读取纯Comma-Separated Values文件,\mintinline{Matlab}{load}适合读取带注释的Comma-Separated Values文件(示例如下),其在读取过程中会自动忽略csv文件的注释。

\begin{minted}{Matlab}
% x                       y                        IsoLevel
-1.348651530446577E-5     1.798983884698175E-5     0.5
-1.4987361701783775E-5    2.4219367655756994E-5    0.5
-1.494145158530118E-5     2.3443649068772022E-5    0.5
... ...
\end{minted}

\section{如何将两个维度不一致的矩阵串联起来?}

现有矩阵\mintinline{Matlab}{a = [1]}和矩阵\mintinline{Matlab}{b = b = [5; 9; 4; 4]},将其横向拼接成一个矩阵\mintinline{Matlab}{c},

\begin{minted}{Matlab}
c = [1, 5;
    1, 9;
    1, 4;
    1, 4]
\end{minted}

思路很简单,因为\mintinline{Matlab}{a}的维度不够,所以将其扩维,然后拼接。

\begin{minted}{Matlab}
a = [1];
b = [5; 9; 4; 4];
[r, ~] = size(b);
c = [repmat(a, r, 1), b];
\end{minted}

\section{颜色的处理}

Hex是一种常用的十六进制颜色码,MATLAB并不能识别,从MathWorks File Exchange找了一个\mintinline{Matlab}{rgb2hex and hex2rgb}的函数\footnote{\url{https://ww2.mathworks.cn/matlabcentral/fileexchange/46289-rgb2hex-and-hex2rgb}},另,MathWorks File Exchange真特么是个宝库,缺什么找什么,一找一个准。

有时候需要将\mintinline{Matlab}{double}类型的矩阵转换成rgb色值的三维矩阵,没有MATLAB内置函数能做到,在MathWorks File Exchange上找了个\mintinline{Matlab}{double2rgb}函数\footnote{\url{https://www.mathworks.com/matlabcentral/fileexchange/30264-double2rgb}}可以很方便的做到这一点。

\section{谈谈MATLAB中的图形对象}

可视化是一项很费时费力的工作,MATLAB可视化成本更高,由于参数混杂,很难进行快速调整。我脑子不好使,先记录一下MATLAB基本图像元素的构成,更详细介绍请查阅MATLAB帮助文档\footnote{\url{https://ww2.mathworks.cn/help/matlab/graphics-objects.html}}。

重点关注两类对象,顶层对象\mintinline{Matlab}{Root, Figure, Axes}和插图对象\mintinline{Matlab}{Colorbar, Legend}。图形对象相关的函数如\autoref{g_o_identification}所示,用于查找、复制和删除图形对象。

\begin{minted}{Matlab}
r = groot;
fig = figure;
ax = gca;
c = colorbar;
lgd = legend('a','b','c');
\end{minted}

\begin{figure}
    \centering
    \includegraphics[height=6cm]{g_o_identification}
    \caption{MATLAB中的图形对象的标识}
    \label{g_o_identification}
\end{figure}

新版本MATLAB推荐使用调用对象的方式修改图形对象属性,老版本用\mintinline{Matlab}{get, set}函数分别查阅和修改对象属性。

\begin{minted}{Matlab}
p = plot(1:10);
p.Color = 'r';
set(p, 'Color', 'red');
\end{minted}

\begin{figure}
    \centering
    \includegraphics[height=6cm]{graphics_objects}
    \caption{MATLAB中的图形对象}
\end{figure}

\subsection{子图}

\mintinline{Matlab}{subplot}函数可以画子图,有两种调用方式,

\begin{minted}{Matlab}
subplot(m,n,p)
subplot('Position',pos)
\end{minted}

第一种方式是创建一个$ m\times n $的网格,并在第$ p $个网格上绘图;第二种方式是在指定\mintinline{Matlab}{pos}上绘图,座标属性为[left bottom width height]。可以通过查阅图像对象来获取图形属性。

\section{函数!函数!}

这里收集几个关于函数的最佳实践,首先是函数的返回值问题。很多时候,MATLAB函数的返回值不止一个,通常让变量一一返回,在外部使用变量一一接收。最佳做法是将需要返回的变量装入\mintinline{Matlab}{struct}中,这样在外部仅使用一个变量就能接收所有返回值。

\section{如何在矩阵中插入一行/一列}

看了一圈,只能通过拼接矩阵来实现,并且MATLAB自己没有这样的函数。自己写了一个。

\begin{minted}{Matlab}
% 按行插入
function mat = row_insert(mat_row, pos, mat_add)
[~, c_row] = size(mat_row);
[~, c_add] = size(mat_add);
if c_row == c_add
    mat = [mat_row(1:pos-1, :); mat_add; mat_row(pos:end, :)];
else
    error('要插入的矩阵列要与原始矩阵一致')
end
end

% 按列插入
function mat = col_insert(mat_row, pos, mat_add)
[r_row, ~] = size(mat_row);
[r_add, ~] = size(mat_add);
if r_row == r_add
    mat = [mat_row(:, 1:pos-1), mat_add, mat_row(:, pos:end)];
else
    error('要插入的矩阵行要与原始矩阵一致')
end
end
\end{minted}

\section{图形处理}

\subsection{基本法}

\subsection{图形插值}

图形插值就是让模糊的图形(像素较少)变成比较清楚的图形(像素变多),稍微科学一点的说法是,图像插值就是利用已知邻近像素点的灰度值(或rgb值)来产生未知像素点的灰度值(或rgb值),以便由原始图像再生出具有更高分辨率的图像。图像处理的学问太多,涉及到的数学也比较深,这里只简单记录应用。

常见的插值算法有最近邻插值算法,双线性插值算法,三次卷积等。具体内容可以去学习一些DIP(Digital image processing)的公开课。如:

\url{http://inside.mines.edu/~whoff/courses/EENG510/lectures/}

\url{http://cs.nju.edu.cn/liyf/dip15/dip.htm}

常用的是MATLAB内置的\mintinline{Matlab}{interp2}函数,

Github上找了个现成的工具箱\footnote{\url{https://github.com/FlorentBrunet/image-interpolation-matlab}},快速实现bicubic和bilinear两种插值算法,效果很不错。

