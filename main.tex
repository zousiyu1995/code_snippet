\documentclass[oneside]{book}

\usepackage{xcolor}

%% 中文文字处理
\colorlet{title}{blue!40!black}
\usepackage[UTF8, heading = true]{ctex}
    \ctexset{today = old}
    \pagestyle{plain}  % 自定义板式
    %% 标题设置
    \ctexset{
        chapter = {
            pagestyle = chapterpage,
        },
        section = {
            name = {问题},
            format = \sffamily\raggedright\color{title}\zihao{4},
            numberformat = \rmfamily,
            number = \arabic{section},
        },
        subsection = {
            format = \sffamily\raggedright\color{title}\zihao{5},
        },
    }%

%% 字体设置
\usepackage{fontspec}
    \setmainfont{TeX Gyre Pagella}
    \setsansfont{TeX Gyre Heros}
    \setmonofont{DejaVu Sans Mono}  % Consolas
    \setCJKmainfont[BoldFont={SimHei},ItalicFont={KaiTi}]{Source Han Serif SC}
    \setCJKmonofont{FangSong}
    \setCJKsansfont{Source Han Sans HW SC}

%% 数学字体
\usepackage{unicode-math}
    \setmathfont[
        math-style = ISO,
        bold-style = ISO
        ]{TeX Gyre Pagella Math}

\usepackage{geometry}
    \geometry{paper=a4paper,
        hmargin=2cm,
        vmargin=2cm,
        marginparwidth=2.5cm,
    }%

%% 代码展示
\definecolor{bg}{rgb}{0.95, 0.95, 0.95}

\usepackage[newfloat=true]{minted}
    \usemintedstyle{colorful}
    \newminted{Matlab}{bgcolor=bg, breaklines=true}  % Matlabcode环境
    \newminted{python}{bgcolor=bg}
    %% 行内代码
    \newmintinline{Matlab}{}

%% 页眉页脚
\usepackage{fancyhdr}
\pagestyle{fancy}
    \fancyhead{}
        \lhead{\kaishu\nouppercase{\leftmark}}
        \rhead{\thepage}
    \fancyfoot{}
% 章标题所在页专用版式
\fancypagestyle{chapterpage}{%
    \chead{\slshape\leftmark}
    \lhead{\sffamily 代码笔记本}
    \rhead{\thepage}
}

%% 脚注,带圈数字,刘海洋,可以超过10
\usepackage{xunicode-addon}
\newfontfamily\fnmarkfont{ipag.ttf} % 带圈 0 到 20 被认做西文符号
\newCJKfontfamily\fnCJKmarkfont{ipag.ttf} % 带圈数字超过 20 是 CJK 符号
\renewcommand\thefootnote{
    {\fnmarkfont\fnCJKmarkfont\textcircled{\arabic{footnote}}}
}

%% 自定义命令

%% 超链接
\usepackage{hyperref}
    \hypersetup{%
        bookmarksopen=true,  % 展开书签
        bookmarksnumbered=true,  % 显示书签编号
        bookmarksopenlevel=1,
        unicode=true,  % 使书签支持unicode字符
        %链接、颜色
        breaklinks=true,  % 链接自动换行
        colorlinks=true,  % 加颜色区分链接
        citecolor=black,  % 文献序号颜色
    }
    %定制pdf属性
    \hypersetup{%
        pdftitle={代码笔记本},
        pdfauthor={邹思宇},
        pdfkeywords={LaTeX, PGFplot, Tikz, MATLAB, C, Python},
        pdfstartview=Fit,%整个页面适合窗口
        pdfcreator={XeLaTeX \& TeXStudio}
    }%


%% url样式
\newfontfamily\urlfont{PT Sans Narrow}
\def\UrlFont{\urlfont}
\urlstyle{urlfont}

\begin{document}
    \frontmatter
    
    \mainmatter
    \chapter*{导言}

这里存放一些实际工作中碰到的实用的代码片段,可能包含MATLAB、Python、C和一些\LaTeX{}的小知识。
    \chapter{MATLAB}

\section{如何遍历当前文件夹及其子文件夹中的全部文件}

假设现在我们有这样一个文件夹A,它含有一些文件和子文件夹B、C、D......我们需要将这个父文件夹(A)及其子文件夹(B、C、D......)中的所有文件名和其路径取出来。

如果你用的是MATLAB 2016b及其后面的版本,那真的太棒了!\Matlabinline|dir()|函数已经支持遍历搜索了。尝试敲入:

\begin{Matlabcode}
dir_data = dir('**/*');
dir_data([dir_data.isdir]) = [];  % 去除所有文件夹
\end{Matlabcode}

这将会返回一个包含文件信息的struct,现在你可以任意操作这些struct了,随意拼接路径。解放大脑,哦也!

方便归方便,但是,一来肯定有大多数人使用的是MATLAB 2016b之前的版本,二来,解放大脑意味着我失去了一次独立思考的机会。

\subsection*{思考}

对于实现方法\footnote{思路来源:\href{https://stackoverflow.com/questions/2652630/how-to-get-all-files-under-a-specific-directory-in-matlab}{How to get all files under a specific directory in MATLAB?}},多层次的遍历,我们第一时间想到的肯定是递归。其次是数据的存储,\Matlabinline|dir()|函数返回的是一个struct,我们要充分利用这个数据结构。所以现在思路是,写一个递归函数,这个函数返回包含所以文件的struct。

这个函数应对先处理父文件夹,获取文件和文件夹,然后获取子文件夹,我们对获取的子文件夹再次调用该函数,最后函数返回存储有所有文件信息的struct。现在,你可以对这个结构体做你想做的事情。

\begin{algorithm}[H]
\KwIn{path}

\For{$ i = 1 $ \KwTo length (sub dir)}{
    get next dir information
    
    递归
}

\KwRet{struct of file information}
\caption{遍历获取当前文件夹及其所有子文件夹中的文件名}
\end{algorithm}

\subsection*{解}

在获取子文件夹的过程中,我们需要去除`.'和`..'这两个特殊的文件夹。

\Matlabfile[firstline=2]{code/get_all_file_name.m}

\subsection*{总结}
\Matlabinline|dir()|函数遍历整个F盘共2万余文件文件大约需要1.555823s。我们实现的递归函数遍历F盘文件大约需要3.703009s。慢是慢了点,但我们成功运用了递归解决问题,不是吗?

\section{title中英文标题}
    \backmatter
\end{document}