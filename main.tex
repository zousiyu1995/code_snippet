\documentclass[oneside]{book}

%% 必须的包
\usepackage{amsmath}
\usepackage{amsfonts}
\usepackage{amssymb}
\usepackage{xcolor}
\usepackage{graphicx}
\graphicspath{{fig/}}
\DeclareGraphicsExtensions{.pdf,.eps,.png,.jpg,.jpeg,.bmp}

%% 中文文字处理
\colorlet{title}{blue!40!black}
\usepackage[heading = true]{ctex}
    \ctexset{
        section = {
            format = \bf\raggedright\zihao{4},
        }
    }

%% 字体设置
\usepackage{fontspec}
    \setmainfont{Adobe Garamond Pro}  % TeX Gyre Pagella
    \setsansfont{Arial}
    \setmonofont{Iosevka Type Slab}  % Source Code Pro, Consolas, DejaVu Sans Mono
    \setCJKmainfont[BoldFont={Source Han Sans SC},
                    ItalicFont={KaiTi}]{Source Han Serif SC}
    \setCJKmonofont{Sarasa Term SC}  % FangSong
    \setCJKsansfont{Source Han Sans SC}

\usepackage{geometry}
    \geometry{paper=a4paper,
%        hmargin = 2cm,
        vmargin = 2cm,
    }%

%% 代码展示
\definecolor{bg}{rgb}{0.95, 0.95, 0.95} % 背景色

\usepackage{minted}  % 启用浮动体支持[newfloat=true]
    \usemintedstyle{colorful}  % 高亮主题
    \setminted{bgcolor=bg, breaklines=true}

\newcommand{\mcode}[1]{\mintinline{Matlab}{#1}}
\newcommand{\pycode}[1]{\mintinline{python}{#1}}
\newcommand{\shcode}[1]{\mintinline{shell}{#1}}

%% 脚注,带圈数字,刘海洋,可以超过10
\usepackage{xunicode-addon}
\newfontfamily\fnmarkfont{ipag.ttf} % 带圈 0 到 20 被认做西文符号
\newCJKfontfamily\fnCJKmarkfont{ipag.ttf} % 带圈数字超过 20 是 CJK 符号
\renewcommand\thefootnote{
    {\fnmarkfont\fnCJKmarkfont\textcircled{\arabic{footnote}}}
}

\usepackage{enumitem}%列表
\setlist[description]{%
    font=\bfseries\color{blue!40!black},
    noitemsep,
    nosep,
    align=left,
    leftmargin=!,
    labelindent=\parindent,
    labelwidth=20mm,
    labelsep=1em,
    itemindent=0mm,
}
\setlist[enumerate]{%
    noitemsep,
    nosep,
    leftmargin=0pt,
    itemindent=*,
    labelindent=\parindent,
    listparindent=\parindent,
}
\setlist[itemize]{%
    nosep,
    leftmargin=\parindent,
}

%% 超链接
\usepackage{hyperref}
    \hypersetup{%
        bookmarksopen=true,  % 展开书签
        bookmarksnumbered=true,  % 显示书签编号
        bookmarksopenlevel=1,
        unicode=true,  % 使书签支持unicode字符
        %链接、颜色
        breaklinks=true,  % 链接自动换行
        colorlinks=true,  % 加颜色区分链接
        linkcolor=title,
        citecolor=black,  % 文献序号颜色
    }
    %定制pdf属性
    \hypersetup{%
        pdftitle={代码笔记本},
        pdfauthor={zousiyu},
        pdfkeywords={LaTeX, PGFplot, Tikz, MATLAB, C, C++, Python, algorithm},
        pdfstartview=Fit,%整个页面适合窗口
        pdfcreator={XeLaTeX \& TeXStudio}
    }%

%自动引用
\AtBeginDocument{%
    \def\figureautorefname{图}%
    \def\tableautorefname{表}%
    \def\partautorefname{卷}%
    \def\appendixautorefname{附录}%
    \def\equationautorefname{式}%
    \def\Itemautorefname{列表}%
    \def\chapterautorefname{章}%
    \def\sectionautorefname{节}%
    \def\subsectionautorefname{小节}%
    \def\subsubsectionautorefname{条目}%
    \def\paragraphautorefname{自然段}%
    \def\Hfootnoteautorefname{脚注}%
    \def\AMSautorefname{式}%
    \def\theoremautorefname{定理}%
    \def\pageautorefname{页}%
}%

%% url样式
\newfontfamily\urlfont{PT Sans Narrow}
\def\UrlFont{\urlfont}
\urlstyle{urlfont}

\begin{document}
    \frontmatter
    \tableofcontents
    \mainmatter
    \chapter*{导言}

这里存放一些实际工作中碰到的实用的代码片段,可能包含MATLAB、Python、C和一些\LaTeX{}的小知识。
    \chapter{MATLAB}

\section{将内置函数背下来}

在MATLAB中工作时,很多操作其实都是有内置函数的,对MATLAB不熟悉就用不成,然后就“曲线救国”了,效率不高且不说,关键自己实现很费脑!这里收集一些数值计算工作中极其常用的函数。

\mintinline{Matlab}{meshgrid}

\section{如何遍历当前文件夹及其子文件夹中的全部文件?}

假设现在我们有这样一个文件夹A,它含有一些文件和子文件夹B、C、D......,这些子文件夹又包含若干层子文件夹。我们需要将这个父文件夹(A)及其子文件夹(B、C、D......)和孙文件夹中的所有文件名和其路径取出来。

如果你用的是MATLAB 2016b及更新的版本,那真的太棒了!\mintinline{Matlab}{dir()}函数已经支持遍历搜索了。尝试敲入:

\begin{minted}{Matlab}
dir_data = dir('**/*');
dir_data([dir_data.isdir]) = [];  % 去除所有.和..文件夹
\end{minted}

这将会返回一个包含文件信息的struct,现在你可以任意操作这些struct了,随意拼接路径。解放大脑,哦也!方便归方便,但是,一来肯定有大多数人使用的是MATLAB 2016b之前的版本,二来,解放大脑意味着我失去了一次独立思考的机会。

\subsection*{思考}

对于实现方法\footnote{思路来源:\href{https://stackoverflow.com/questions/2652630/how-to-get-all-files-under-a-specific-directory-in-matlab}{How to get all files under a specific directory in MATLAB?}},多层次的遍历,我第一时间想到的是递归。然后就是数据的存储了,\mintinline{Matlab}{dir()}函数返回的是一个struct,这个数据结构储存有文件的信息,我们要充分利用这个数据结构。所以现在思路是,写一个递归函数,这个函数返回包含所有文件信息的struct。

这个函数应对先处理父文件夹,获取文件和子文件夹,然后储存文件信息,同时去除子文件夹中的`.'和`..'这两个特殊文件夹。我们对获取的子文件夹再次调用该函数,并储存文件信息。如此,利用递归获取子子孙孙无穷尽文件夹的信息\footnote{其实这并不可能,因为递归是有栈高度限制的,调用函数压入栈,返回函数弹出栈,如果文件夹层次太深,一直压栈就会到达栈溢出警告的极限,例如Python的栈往往是100层,我想MATLAB的栈也大致如此,不会太高},最后函数返回存储有所有文件信息的struct。现在,你可以对这个结构体做你想做的事情。

\subsection*{解}

MATLAB 2016b以上的版本我们可以用函数返回struct,这个数据结构包含[folder, name, date, bytes, isdir, datenum]六个字段的信息,我们可以按自己意愿使用folder和name拼接出文件的完整路径。

\inputminted[firstline=1]{Matlab}{code/matlab/get_all_file_name_R2016b_newer.m}

MATLAB 2016a及之前的版本dir struct信息并不包含folder,如果返回struct,将只有文件的[name, date, bytes, isdir, datenum]五个字段的信息,所以我们并不能根据函数返回的struct拼接出文件完整路径,\textbf{我们需要自己将路径拼接成一个cell,然后使用函数返回cell}。

\inputminted{Matlab}{code/matlab/get_all_file_name_R2016a_older.m}

\subsection*{总结}
\mintinline{Matlab}{dir()}函数遍历整个F盘共2万余文件文件大约需要1.555823s。我们实现的递归函数遍历F盘文件大约需要3.703009s。慢是慢了点,但我们成功运用了递归解决问题,不是吗?

\section{如何按自然顺序排序字符串?}

通常,我们会遇到处理一系列文件名有规律的文件的情况,比如:a1.txt、a2.txt ...... a100.txt。但是,当读取文件名到一个cell里后,我们发现文件名往往是乱序排列的,甚至当你使用\mintinline{Matlab}{sort}函数后,排序也不会改变。搜索了一下,在Mathworks File Exchange网站找到了一个自然排序的函数\footnote{\url{https://cn.mathworks.com/matlabcentral/fileexchange/34464-customizable-natural-order-sort}},感谢作者Stephen Cobeldick。效果如下:

\begin{figure}[h]
    \centering
    \begin{minipage}{0.45\textwidth}
        \centering
        \includegraphics[height=6cm]{disordered_file_name}
        \caption{乱序的文件名}
    \end{minipage}
    \begin{minipage}{0.45\textwidth}
        \centering
        \includegraphics[height=6cm]{ordered_file_name}
        \caption{排序后自然顺序的文件名}
    \end{minipage}
\end{figure}

\section{如何隔行取数据?}

闭上眼睛,想象现在有这样一个数组\mintinline{Matlab}{[1, 2, 3, 4, 5, 6, 7, 8, 9, 10]},我们要隔一列取一个数据,或者隔两列取一个数据。得益于MATLAB的向量化编程,我们可以很方便的做到,

\begin{minted}{Matlab}
mat_a = [1, 2, 3, 4, 5, 6, 7, 8, 9, 10];
mat_b = mat_a(:, 1:2:length(mat_a));
\end{minted}

如果你用循环,那么你的代码就不优雅,另,向量化操作比循环快,大型数组优势明显。以上。

\section{如何在遍历数组的同时删除被遍历过的元素?}

闭上眼睛,想象现在有这样一个数组\mintinline{Matlab}{[1, 2, 3, 4, 5, 6, 7, 8, 9, 10]},我们需要边遍历元素边删除元素。实现方法和Python章节方法一致。

\begin{minted}{Matlab}
mat_a = [1, 2, 3, 4, 5, 6, 7, 8, 9, 10];

while ~isempty(mat_a)
    fprintf("The element being traversed is %d\n", mat_a(1));
    mat_a(1) = [];
    disp(mat_a);
end
\end{minted}

\section{再谈向量化操作}

今天又碰到一个数组操作的问题,同样,如果用一般的方法来解决,代码是很冗长的,向量化操作再次助我一臂之力。

有一个2列的数组\mintinline{Matlab}{all_data},第一列有正有负,我们称第一列为$ x $,第二列为$ y $。现在需要索引$ x>0 $时对应的$ [x, y] $为一个新的数组\mintinline{Matlab}{a}。并且需要从\mintinline{Matlab}{a}中返回$ y=\min(y) $时所对应的数组$ [x_0, y_0] $。

\begin{minted}{Matlab}
... ...
-1.44319267634370e-06 9.80637785912817e-06
-1.68967863180042e-07 9.73806551956721e-06
-6.45218837777561e-07 9.75074561060079e-06
6.28923217787410e-07 9.75037059307950e-06
1.54045071473931e-07 9.73772289244816e-06
1.42591401762642e-06 9.80510552313044e-06
... ...
\end{minted}

先谈向量化获取数组\mintinline{Matlab}{a},利用逻辑索引,保证$ x $全大于0,并取出1、2两列;然后利用\mintinline{Matlab}{find}获取$ y=\min(y) $的行索引;最后利用索引轻松找到需要的数据。可能看起来比较难理解,但是此时再在外面套文件操作的循环等循环操作是不是清晰多了。

\begin{minted}{Matlab}
a = all_data(all_data(:, 1) > 0, 1:2);

[r, ~] = find(a(:, 2) == min(a(:, 2)));
what_is_i_need = a(r, c);
\end{minted}

\begin{minted}{Matlab}
clear
clc

file_name_struct = dir('./0518*.txt');
file_name = {file_name_struct(:).name};
file_name = natsort(file_name);
what_is_i_need = [];

for file_num = 1:length(file_name)
    all_data = load(file_name{file_num});
    a = all_data(all_data(:, 1) > 0, 1:2);
    
    [r, ~] = find(a(:, 2) == min(a(:, 2)));  % 
    what_is_i_need = [what_is_i_need; a(r, 1), a(r, 2)];
end

plot(what_is_i_need(:,2))
\end{minted}

再来记录一个问题:循环操作里面有一个的2列数组\mintinline{Matlab}{all_data},每次循环取第一列中与0.5最接近的数据和对应的列,所以该数组大小会不断变,设其维度为$ n\times 2 $。如果$ n<2 $,我们将数据置为0并保存到一个新数组里面去;如果$ n>=2 $,保存其最小值和最大值到新数组里面去。同样,利用向量化操作最大程度减少代码量。

\begin{minted}{Matlab}
pos = [];
for i = 1:length(time)
    % [x, phi]
    all_data = load(strcat(mph_file, '\', num2str(i), '.txt'));
    all_data = all_data(abs(all_data(:, 2)-0.5) < 0.01, :);
    [r, c] = size(all_data);
    if r == 0 || r == 1
        x_min = 0;
        x_max = 0;
        phi = 0;
        pos = [pos; time(i), x_min, phi; time(i), x_max, phi];
    else
        x_min = all_data(all_data(:, 1) == min(all_data(:, 1)));
        x_max = all_data(all_data(:, 1) == max(all_data(:, 1)));
        [r, ~] = find(all_data == x_min);
        phi_min = all_data(r, 2);
        [r, ~] = find(all_data == x_max);
        phi_max = all_data(r, 2);
        pos = [pos; time(i), x_min, phi_min; time(i), x_max, phi_max];
    end
end
\end{minted}

\section{文件读取}

\mintinline{Matlab}{csvread}适合读取纯Comma-Separated Values文件,\mintinline{Matlab}{load}适合读取带注释的Comma-Separated Values文件(示例如下),其在读取过程中会自动忽略csv文件的注释。

\begin{minted}{Matlab}
% x                       y                        IsoLevel
-1.348651530446577E-5     1.798983884698175E-5     0.5
-1.4987361701783775E-5    2.4219367655756994E-5    0.5
-1.494145158530118E-5     2.3443649068772022E-5    0.5
... ...
\end{minted}

\section{如何将两个维度不一致的矩阵串联起来?}

现有矩阵\mintinline{Matlab}{a = [1]}和矩阵\mintinline{Matlab}{b = b = [5; 9; 4; 4]},将其横向拼接成一个矩阵\mintinline{Matlab}{c},

\begin{minted}{Matlab}
c = [1, 5;
    1, 9;
    1, 4;
    1, 4]
\end{minted}

思路很简单,因为\mintinline{Matlab}{a}的维度不够,所以将其扩维,然后拼接。

\begin{minted}{Matlab}
a = [1];
b = [5; 9; 4; 4];
[r, ~] = size(b);
c = [repmat(a, r, 1), b];
\end{minted}

\section{颜色的处理}

Hex是一种常用的十六进制颜色码,MATLAB并不能识别,从MathWorks File Exchange找了一个\mintinline{Matlab}{rgb2hex and hex2rgb}的函数\footnote{\url{https://ww2.mathworks.cn/matlabcentral/fileexchange/46289-rgb2hex-and-hex2rgb}},另,MathWorks File Exchange真特么是个宝库,缺什么找什么,一找一个准。

有时候需要将\mintinline{Matlab}{double}类型的矩阵转换成rgb色值的三维矩阵,没有MATLAB内置函数能做到,在MathWorks File Exchange上找了个\mintinline{Matlab}{double2rgb}函数\footnote{\url{https://www.mathworks.com/matlabcentral/fileexchange/30264-double2rgb}}可以很方便的做到这一点。

\section{谈谈MATLAB中的图形对象}

可视化是一项很费时费力的工作,MATLAB可视化成本更高,由于参数混杂,很难进行快速调整。我脑子不好使,先记录一下MATLAB基本图像元素的构成,更详细介绍请查阅MATLAB帮助文档\footnote{\url{https://ww2.mathworks.cn/help/matlab/graphics-objects.html}}。

重点关注两类对象,顶层对象\mintinline{Matlab}{Root, Figure, Axes}和插图对象\mintinline{Matlab}{Colorbar, Legend}。图形对象相关的函数如\autoref{g_o_identification}所示,用于查找、复制和删除图形对象。

\begin{minted}{Matlab}
r = groot;
fig = figure;
ax = gca;
c = colorbar;
lgd = legend('a','b','c');
\end{minted}

\begin{figure}
    \centering
    \includegraphics[height=6cm]{g_o_identification}
    \caption{MATLAB中的图形对象的标识}
    \label{g_o_identification}
\end{figure}

新版本MATLAB推荐使用调用对象的方式修改图形对象属性,老版本用\mintinline{Matlab}{get, set}函数分别查阅和修改对象属性。

\begin{minted}{Matlab}
p = plot(1:10);
p.Color = 'r';
set(p, 'Color', 'red');
\end{minted}

\begin{figure}
    \centering
    \includegraphics[height=6cm]{graphics_objects}
    \caption{MATLAB中的图形对象}
\end{figure}

\subsection{子图}

\mintinline{Matlab}{subplot}函数可以画子图,有两种调用方式,

\begin{minted}{Matlab}
subplot(m,n,p)
subplot('Position',pos)
\end{minted}

第一种方式是创建一个$ m\times n $的网格,并在第$ p $个网格上绘图;第二种方式是在指定\mintinline{Matlab}{pos}上绘图,座标属性为[left bottom width height]。可以通过查阅图像对象来获取图形属性。

\section{函数!函数!}

这里收集几个关于函数的最佳实践,首先是函数的返回值问题。很多时候,MATLAB函数的返回值不止一个,通常让变量一一返回,在外部使用变量一一接收。最佳做法是将需要返回的变量装入\mintinline{Matlab}{struct}中,这样在外部仅使用一个变量就能接收所有返回值。

\section{如何在矩阵中插入一行/一列}

看了一圈,只能通过拼接矩阵来实现,并且MATLAB自己没有这样的函数。自己写了一个。

\begin{minted}{Matlab}
% 按行插入
function mat = row_insert(mat_row, pos, mat_add)
[~, c_row] = size(mat_row);
[~, c_add] = size(mat_add);
if c_row == c_add
    mat = [mat_row(1:pos-1, :); mat_add; mat_row(pos:end, :)];
else
    error('要插入的矩阵列要与原始矩阵一致')
end
end

% 按列插入
function mat = col_insert(mat_row, pos, mat_add)
[r_row, ~] = size(mat_row);
[r_add, ~] = size(mat_add);
if r_row == r_add
    mat = [mat_row(:, 1:pos-1), mat_add, mat_row(:, pos:end)];
else
    error('要插入的矩阵行要与原始矩阵一致')
end
end
\end{minted}

\section{图形处理}

\subsection{基本法}

\subsection{图形插值}

图形插值就是让模糊的图形(像素较少)变成比较清楚的图形(像素变多),稍微科学一点的说法是,图像插值就是利用已知邻近像素点的灰度值(或rgb值)来产生未知像素点的灰度值(或rgb值),以便由原始图像再生出具有更高分辨率的图像。图像处理的学问太多,涉及到的数学也比较深,这里只简单记录应用。

常见的插值算法有最近邻插值算法,双线性插值算法,三次卷积等。具体内容可以去学习一些DIP(Digital image processing)的公开课。如:

\url{http://inside.mines.edu/~whoff/courses/EENG510/lectures/}

\url{http://cs.nju.edu.cn/liyf/dip15/dip.htm}

常用的是MATLAB内置的\mintinline{Matlab}{interp2}函数,

Github上找了个现成的工具箱\footnote{\url{https://github.com/FlorentBrunet/image-interpolation-matlab}},快速实现bicubic和bilinear两种插值算法,效果很不错。


    \chapter{Python}

\section{如何展开一个嵌套的序列?}

我们现在有这样一个序列\codeinline{python}{items = [1, 2, [3, 4, [5, 6, [9, 8], 7], 8]]},我们想逐级展开这个序列,然后将所有元素装入一个序列。

如果这个序列层级较少,我们可以用多层\codeinline{python}{for}循环来遍历这个序列。一旦这个序列超过3层,过多的循环会让你很头疼。同样,这种多层级的问题我们可以用\textbf{递归}来解决。构建一个函数,这个函数能处理第一层的元素,由于第二层是\codeinline{python}{list},它是一个可迭代对象,我们只需要判断第二层是不是可迭代对象,同时忽略\codeinline{python}{str, bytes}对象\footnote{\codeinline{python}{str, bytes}也是可迭代对象,我们要避免其展开成单个字符。}。只要内层是可迭代的,我们就开始递归,对其应用该函数。

\pythonfile{code/python/unfold.py}

由于存在递归,所以函数会被调用很多次,每次调用所得的数据都需要保留,如何在多次的调用之间共享保留数据呢?我采用一个默认参数来实现,首次调用时不给默认参数新值,这会产生一个空的\codeinline{python}{list},当对内层对象调用时,将上一次产生的数据赋值给这个参数。输出结果:

\begin{pythoncode}
>>> items1 = ['Paula', ['Thomas', 'Lewis', ['siyu', 'ziyan', ['jianyuan']]]]
>>> items2 = [1, 2, [3, 4, [5, 6, [9, 8], 7], 8]]
>>> items3 = [[1, 2], 3, (4, [5, 6])]
>>> print(unfold(items1))
>>> print(unfold(items2))
>>> print(unfold(items3))
['Paula', 'Thomas', 'Lewis', 'siyu', 'ziyan', 'jianyuan']
[1, 2, 3, 4, 5, 6, 9, 8, 7, 8]
[1, 2, 3, 4, 5, 6]
\end{pythoncode}

但这样做有两个显而易见的坏处,一是当我们的嵌套序列有无限多层,递归会栈溢出;二是序列整个被读取到内存中了,当序列元素非常多,比如1亿,内存会被撑死。坏处一我们不去管他,大多数情况下是适用的,坏处二可以很容易的利用generator来解决\footnote{思路来源~\url{http://python3-cookbook.readthedocs.io/zh_CN/latest/c04/p14_flattening_nested_sequence.html}}。

\pythonfile{code/python/unfold_generator.py}

使用generator一来能防止内存爆炸,二来不需要在函数的多次调用见传递数据,代码更清晰明朗。需要注意,generator是惰性序列,边调用边计算,我们需要使用\codeinline{python}{for}迭代出每一个元素或者直接用\codeinline{python}{list()}获取全部元素。

\begin{pythoncode}
items1 = ['Paula', ['Thomas', 'Lewis', ['siyu', 'ziyan', ['jianyuan']]]]
items2 = [1, 2, [3, 4, [5, 6, [9, 8], 7], 8]]
items3 = [[1, 2], 3, (4, [5, 6])]
print(list(unfold(items1)))
print(list(unfold(items2)))
print(list(unfold(items3)))
\end{pythoncode}

\section{如何遍历当前文件夹及其子文件夹中的全部文件}

前面用MATLAB实现了一个,现在用Python来实现。第一种方法是利用递归来实现,思路同样是先找文件,然后找子文件夹,最后对子文件夹递归;第二种方法是利用os.walk模块,并将其做成generator,这样在应对大量的文件时会有优势。推荐第二种方法,一来os模块考虑了很多我们忽略了的细节\footnote{比如,如果递归版本的函数遍历的根目录是一个磁盘,这个磁盘上的特殊的文件夹“System Volume Information”又是禁止被访问的,这时就会抛出一个PermissionError。笨一点的解决办法是从子目录的list中删除这个目录,好一点的办法就是用os模块了。},二来generator是一个优雅的设计,用Python就应该好好学用generator。

\pythonfile{code/python/get_all_file_name.py}



    \chapter{C和C++}

\section{C语言的动态数组}

大多数时候为了方便(其实是我菜),会使用库较多(方便)的C++,但是C语言在实际生产中使用率仍然很高,比如长期使用的ANSYS Fluent的UDF就不得不用C语言。下面是一个简易的动态数组的实现,来源\footnote{\url{https://stackoverflow.com/questions/3536153/c-dynamically-growing-array}}。

\begin{minted}{c}
#include <iostream>

typedef struct
{
    int *array;
    size_t used;
    size_t size;
} Array;

void initArray(Array *a, size_t initialSize)
{
    a->array = (int *)malloc(initialSize * sizeof(int));
    a->used = 0;
    a->size = initialSize;
}

void insertArray(Array *a, int element)
{
    // a->used is the number of used entries, because a->array[a->used++] updates a->used only *after* the array has been accessed.
    // Therefore a->used can go up to a->size
    if (a->used == a->size)
    {
        a->size *= 2;
        a->array = (int *)realloc(a->array, a->size * sizeof(int));
    }
    a->array[a->used++] = element;
}

void freeArray(Array *a)
{
    free(a->array);
    a->array = NULL;
    a->used = a->size = 0;
}

int main()
{
    Array a;
    int i;

    initArray(&a, 5); // initially 5 elements
    for (i = 0; i < 100; i++)
        insertArray(&a, i);     // automatically resizes as necessary
    printf("%d\n", a.array[9]); // print 10th element
    printf("%d\n", a.used);     // print number of elements
    freeArray(&a);
    return 0;
}

\end{minted}
    \chapter{算法}

\section{排序}

先来了解几个基本概念:

排序:将一组“无序”的记录序列调整为“有序”的记录序列。

稳定性:假定在待排序的记录序列中,存在多个具有相同的关键字的记录,若经过排序,这些记录的相对次序保持不变,则称这种排序算法是稳定的,否则称为不稳定的。

排序算法的分类:插入类、交换类、选择类、归并类和基数类。

基于⽐较的排序算法的最佳性能为$ O(n\log n) $。
    \chapter{Git}

\section{如何给Git仓库添加一个空文件夹?}
默认情况下,空文件夹不被记录,也不能被推送。特殊需求参见\href{https://stackoverflow.com/questions/115983/how-can-i-add-an-empty-directory-to-a-git-repository}{How can I add an empty directory to a Git repository? - Stack Overflow}


    \backmatter
\end{document}